%%%%%%%%%%%%%%%%%%%%%%%%%%%%%%%%%%%%%%%%%%%%%%%%%%%%%%%%%%%%
%%  This Beamer template was created by Cameron Bracken.
%%  Anyone can freely use or modify it for any purpose
%%  without attribution.
%%
%%  The current presentation created by Jeferson L. R. Souza (jefecomp) is based on the template created by Cameron Bracken. 
%%  
%%  Small modifications have been introduced and anyone is free to use such modified version.
%%
%% Last Modified: June 14, 2015.

\documentclass[xcolor=x11names,compress]{beamer}

%% General document %%%%%%%%%%%%%%%%%%%%%%%%%%%%%%%%%%
\usepackage{graphicx}
\usepackage{tikz}
\usetikzlibrary{decorations.fractals}
%%%%%%%%%%%%%%%%%%%%%%%%%%%%%%%%%%%%%%%%%%%%%%%%%%%%%%

%Hyperref
\usepackage{hyperref}

%Multirow package
\usepackage{multirow} 

%Math packages
\usepackage{amsmath}
\usepackage{textcomp}


%% Beamer Layout %%%%%%%%%%%%%%%%%%%%%%%%%%%%%%%%%%
\useoutertheme[footline=authorinstitutetitle,subsection=false,shadow]{miniframes}
\useinnertheme{default}
\usefonttheme{professionalfonts}
\usepackage{palatino}

\setbeamerfont{title like}{shape=\scshape,series=\bfseries}
\setbeamerfont{frametitle}{shape=\scshape,series=\bfseries}

\setbeamercolor*{lower separation line head}{bg=Green3} 
\setbeamercolor*{upper separation line foot}{bg=Green3} 
\setbeamercolor*{normal text}{fg=black,bg=white} 
\setbeamercolor*{alerted text}{fg=black,bg=black!10} 
\setbeamercolor*{example text}{fg=black} 
\setbeamercolor*{structure}{fg=black}
 
\setbeamercolor*{palette tertiary}{fg=black,bg=black!3} 
\setbeamercolor*{palette quaternary}{fg=black,bg=black!10} 

%%%%%%%%%%%%%%%%%%%%%%%%%%%%%%%%%%%%%%%%%%%%%%%%%%

\setbeamertemplate{blocks}[rounded] [shadow=true]
\setbeamertemplate{frametitle continuation}[from second][(Continuação)]

%%  declaring picture extensions and default path
\DeclareGraphicsExtensions{.png, .jpg, .pdf}
\graphicspath{{pictures/}}

%% Supporting source code lists
\usepackage{listings}
\lstset{breakatwhitespace,
language=Java,
columns=fullflexible,
keepspaces,
breaklines,
tabsize=3, 
showstringspaces=false,
extendedchars=true}

%Text position
\usepackage{textpos}
\setlength{\TPHorizModule}{128mm}
\setlength{\TPVertModule}{96mm}

\usepackage{array}

%Puting text and other float elements over pictures
\usepackage[percent]{overpic}


%% Hyperlinks over all the document
\usepackage{hyperref}

%% Controlling text alignment
\usepackage{ragged2e}

%% Framed text
\usepackage{framed}

%% Math packages
\usepackage{amsmath}

\begin{document}

\title[Introdução à Gerência de Projetos \hskip20mm \insertframenumber / \inserttotalframenumber  \hskip33mm \inserttitlegraphic]{Introdução à Gerência de Projetos \\[4mm]
\includegraphics[keepaspectratio,width=.4\textwidth]{project-management-2738521_1920}}
\author[@2018 Prof. Jeferson Souza, MSc (jefecomp) - All rights reserved.]{
	\textcolor{blue}{Prof. Jeferson Souza, MSc.} \\[1mm] 
	\textcolor{blue}{\textit{{\footnotesize (jefecomp) }}}\\[1.5mm]
	 \underline{{\footnotesize jeferson.souza@udesc.br}}
	 \vspace*{1mm}
}
\institute[]{\centering \includegraphics[keepaspectratio,width=.5\textwidth]{template/logo_udesc_joinville_horizontal_assinatura}}

\date{}

\titlegraphic{\includegraphics[keepaspectratio,width=.2\textwidth]{template/logo_udesc_joinville_horizontal_assinatura}}

%%%%%%%%%%%%%%%%%%%%%%%%%%%%%%%%%%%%%%%%%%%%%%%%%%%%%%
%%%%%%%%%%%%%%%%%%%%%%%%%%%%%%%%%%%%%%%%%%%%%%%%%%%%%%
\begin{frame}[plain,noframenumbering]
\titlepage
\end{frame}

%%%%%%%%%%%%%%%%%%%%%%%%%%%%%%%%%%%%%%%%%%%%%%%%%%%%%%
%%%%%%%%%%%%%%%%%%%%%%%%%%%%%%%%%%%%%%%%%%%%%%%%%%%%%%
\section{Conceitos}
\subsection{Conceitos}
\begin{frame}{O que é um Projeto?}

\begin{itemize}
\itemsep 5mm

\item Esforço temporário empreendido para criar um produto, serviço, ou resultado exclusivo;

\item O termo temporário significa que todo projeto tem um início e um fim bem definidos;

\item Um projeto termina quando:
\begin{itemize}
\itemsep 3mm
\item Os objetivos foram atingidos;

\item Os objetivos não poderão ser mais atingidos;

\item O projeto não for mais necessário.

\end{itemize}
\end{itemize}
\end{frame}

\begin{frame}[allowframebreaks=.5]{Resultados de Um Projeto}

\begin{itemize}
\itemsep 5mm

\item O resultado de um projeto (normalmente) não é temporário;

\item Um produto (software) resultante de um projeto é desenvolvido para ser utilizado por bastante tempo;

\item Os projetos podem criar:
\begin{itemize}
\itemsep 3mm

\item Um produto final ou um componente;

\item Um serviço;

\item Uma contribuição científica.

\end{itemize}

\end{itemize}

\end{frame}

\begin{frame}{Motivos da Criação de Projetos}

\begin{itemize}
\itemsep 5mm

\item Demanda de mercado;

\item Necessidade organizacional;

\item Solicitação de um cliente (projeto customizado);

\item Avanço tecnológico;

\item Requisito legal.

\end{itemize}

\end{frame}

%%%%%%%%%%%%%%%%%%%%%%%%%%%%%%%%%%%%%%%%%%%%%%%%%%%%%%
%%%%%%%%%%%%%%%%%%%%%%%%%%%%%%%%%%%%%%%%%%%%%%%%%%%%%%
\section{Gerência}
\subsection{Gerência}

\begin{frame}{O que é Gerência de Projetos?}

\begin{block}{}

É a aplicação do conhecimento habilidades, ferramentas, e técnicas para a gestão das atividades do projeto, com o objetivo de atender seus requisitos.
\end{block}

\pause

\begin{block}{}
Envolve o planejamento, monitoramento, e controle de pessoas, processos, e eventos relacionados ao desenvolvimento do projeto.
\end{block}

\end{frame}

\begin{frame}{O que é Gerência de Projetos? (Continuação)}

Gerenciar um projeto inclui:

\begin{itemize}
\itemsep 5mm

\item Identificar as necessidades;

\item Estabelecer objetivos claros e alcançáveis;

\item Equilibrar os conflitos existentes entre qualidade, escopo, tempo, e custo;

\item Adaptar as especificações, os planos, e as abordagens às diferentes preocupações e expectativas das diversas partes envolvidas.

\end{itemize}

\end{frame}

\begin{frame}{Qual o papel do Gerente de Projetos?}

\begin{block}{}

O gerente de projetos é a pessoa responsável pela realização dos objetivos do projeto.
\end{block}

\end{frame}

\begin{frame}[allowframebreaks=.8]{Desenvolvimento de Software Sem Gerência}

\begin{itemize}
\itemsep 5mm

\item É improvisado;

\item Não é rigorosamente seguido;

\item É altamente dependente dos profissionais;

\item A visão de progresso e da qualidade do produto é baixa;

\item A qualidade do produto é definida em função dos prazos.

\end{itemize}

Uma organização sem gerência no desenvolvimento pode ter características tais como:

\begin{itemize}
\itemsep 5mm

\item Reacionária;

\item Cronogramas e orçamentos extrapolados com frequência;

\item Prazos curtos, datas urgentes, e qualidade baixa;

\item Atividades de revisão e testes encurtadas ou eliminadas.

\end{itemize}

\end{frame}

\begin{frame}{Necessidades de Gerência de Projetos de Software}

\begin{block}{}
Desenvolver software é uma atividade complexa.
\end{block}

\pause

\begin{block}{}
Manter pessoas trabalhando corretamente durante um período de tempo também é uma atividade complexa.
\end{block}

\pause

\begin{alertblock}{\centering Então ...}
É exatamente para gerir todas essas complexidades que é necessário que um projeto seja gerenciado.
\end{alertblock}

\end{frame}

\begin{frame}{Importância da Gerência de Projetos}

\begin{itemize}
\itemsep 5mm

\item Garantir qualidade do produto final;

\item Ter segurança para lidar com mudanças sofridas ao longo do ciclo de desenvolvimento do projeto;

\item Organizar profissionais de forma a maximizar o seu rendimento.

\end{itemize}

\end{frame}

%%%%%%%%%%%%%%%%%%%%%%%%%%%%%%%%%%%%%%%%%%%%%%%%%%%%%%
%%%%%%%%%%%%%%%%%%%%%%%%%%%%%%%%%%%%%%%%%%%%%%%%%%%%%%
\section{Os Quatro Ps}
\subsection{Os Quatro Ps}
\begin{frame}{Os Quatro Ps}

A gerência de projetos afeta diretamente quatro principais fatores:

\begin{itemize}
\itemsep 5mm

\item Pessoas;

\item Produto;

\item Processo;

\item Projeto.
\end{itemize}

\begin{block}{}
\centering Esses quatro fatores são denominados de quatro Ps.
\end{block}
\end{frame}

\begin{frame}{\underline{P}essoas}

\begin{itemize}
\itemsep 5mm

\item Considerado o fator mais importante para o sucesso de uma projeto (de software);

\item A gestão de pessoas inclui: recrutamento, seleção, treinamento, remuneração, desenvolvimento de carreira, projeto do trabalho, e desenvolvimento da equipe.

\end{itemize}

\end{frame}

\begin{frame}{\underline{P}roduto}

\begin{itemize}
\itemsep 5mm

\item Devem ser definidos o escopo e os objetivos de um produto, antes de planejar efetivamente o projeto;

\item A gestão de produto inclui as atividades de engenharia de processos de negócio e engenharia de requisitos.

\end{itemize}

\end{frame}

\begin{frame}{\underline{P}rocesso}

\begin{itemize}
\itemsep 5mm

\item Fornecer suporte para a especificação do plano de trabalho;

\item Inclui as atividades de fundamentais e as atividades complementares do planejamento e gerência de projetos.

\end{itemize}

\end{frame}

\begin{frame}{Projeto}

\begin{itemize}
\itemsep 5mm

\item O que se deve gerir;

\item Importante entender os fatores críticos dos processos que envolvem o projeto;

\item Necessário atividades de planejamento, monitoramento, e controle do projeto.

\end{itemize}


\end{frame}

%%%%%%%%%%%%%%%%%%%%%%%%%%%%%%%%%%%%%%%%%%%%%%%%%%%%%%
%%%%%%%%%%%%%%%%%%%%%%%%%%%%%%%%%%%%%%%%%%%%%%%%%%%%%%
\section{Papeis}
\subsection{Papeis}

\begin{frame}{Papéis Dentro de Um Projeto de Software}

\begin{itemize}
\itemsep 5mm

\item Gerentes seniores;

\item Gerentes de projeto;

\item Engenheiros de Software, Analistas de negócio, Analistas de testes, entre outros;

\item Clientes;

\item Usuários finais.

\end{itemize}

\end{frame}

\begin{frame}{Papéis Dentro de Um Projeto de Software}

\begin{alertblock}{Gerentes seniores:}
Definem aspectos do negócio que tem influência sobre o projeto.
\end{alertblock}

\pause

\begin{alertblock}{Gerentes de projeto:}
Devem planejar, motivar, organizar e controlar os profissionais técnicos, e interagir com o nível de gerência para justificar o andamento do projeto. Além disso tem que estar atentos as necessidades de clientes e usuários finais para evitar imprevistos e viabilidades técnicas que comprometam o planejamento.
\end{alertblock}

\end{frame}

\begin{frame}{Papéis Dentro de Um Projeto de Software}

\begin{alertblock}{Profissionais (Egenheiros de Software, Analistas, etc):}
Fornecem as aptidões técnicas necessárias para concretizar o projeto;
\end{alertblock}

\begin{alertblock}{Clientes:}
Especificam os requisitos do que deverá ser construído.
\end{alertblock}

\begin{alertblock}{Usuários Finais}
Pessoas que irão interagir com o software depois que o mesmo estiver pronto para o uso.
\end{alertblock}
\end{frame}

%%%%%%%%%%%%%%%%%%%%%%%%%%%%%%%%%%%%%%%%%%%%%%%%%%%%%%
%%%%%%%%%%%%%%%%%%%%%%%%%%%%%%%%%%%%%%%%%%%%%%%%%%%%%%
\section{Equipes e Paradigmas}
\subsection{Equipes}

\begin{frame}{Equipes}

A estrutura de uma equipe para execução de um projeto, depende de diversos fatores, tais como:

\begin{itemize}
\itemsep 5mm

\item Dificuldade do problema;

\item Tamanho do problema;

\item Período que a equipe ficará junta;

\item Grau de modularização;

\item Qualidade e confiabilidade exigidas pelo sistema;

\item O grau de comunicação exigido pelo projeto.

\end{itemize}

\end{frame}

\begin{frame}{Paradigmas para Estruturar Equipes}

Existem 3 paradigmas genéricos que podem ser adotados na estruturação de equipes (Classificação de Mantei~\cite{Pressman-2001}):

\begin{itemize}

\itemsep 5mm

\item Paradigma democrático e descentralizado;

\item Paradigma controlado e descentralizado;

\item Paradigma controlado e centralizado.

\end{itemize}

\end{frame}

\begin{frame}{Paradigma Democrático e Descentralizado}

\begin{itemize}
\itemsep 5mm

\item Não existe a definição de um líder de equipe permanente;

\item Decisões no projeto são realizadas em grupo e baseadas em consenso;

\item Comunicação entre os membros é horizontal.

\end{itemize}

\end{frame}

\begin{frame}{Paradigma Controlado e Descentralizado}

\begin{itemize}
\itemsep 5mm

\item Define líderes e sublíderes dentro do projeto;

\item Decisões no projeto são realizadas dentro dos grupos de atividades tendo abrangência do líder e sublíderes;

\item Comunicação entre os membros de um sub grupo é horizontal;

\item Comunicações verticais podem ocorrer entre líderes e sublíderes de um grupo de atividades.

\end{itemize}

\end{frame}

\begin{frame}{Paradigma Controlado e Centralizado}

\begin{itemize}
\itemsep 5mm

\item Define um líder que é o centralizador das decisões da equipe;

\item Líder gerencia os demais membros da equipe;

\item Comunicação entre os membros e líder é vertical.

\end{itemize}

\end{frame}

\begin{frame}{Paradigmas para Estruturar Equipes II}

Além da classificação de Mantei, temos ainda a classificação de Constantine que define quatro paradigmas~\cite{Pressman-2001}:

\begin{itemize}
\itemsep 5mm

\item Paradigma fechado;

\item Paradigma aberto;

\item Paradigma aleatório;

\item Paradigma síncrono;

\end{itemize}

\end{frame}

\begin{frame}{Paradigma Fechado}

\begin{itemize}
\itemsep 5mm

\item Hierarquia tradicional de autoridades;

\item Funciona bem para produzir software semelhante a anteriores, porém  não permite muita inovação.

\end{itemize}

\end{frame}

\begin{frame}{Paradigma Aberto}

\begin{itemize}
\itemsep 5mm

\item Trabalho é realizado com intensa colaboração;

\item Decisões baseadas em consenso;

\item Adequado a solução de problemas complexos.

\end{itemize}

\end{frame}

\begin{frame}{Paradigma Aleatório}

\begin{itemize}
\itemsep 5mm

\item Equipe fracamente estruturada, dependendo diretamente da iniciativa individual;

\item Adequado para casos de inovação, porém não é muito eficiente quando um desempenho ordenado é requerido.

\end{itemize}

\end{frame}

\begin{frame}{Paradigma Síncrono}

\begin{itemize}
\itemsep 5mm

\item Segmenta o problema entre os membros da equipe, permitindo que cada um desenvolva uma parte sem muita comunicação.

\end{itemize}

\end{frame}

%%%%%%%%%%%%%%%%%%%%%%%%%%%%%%%%%%%%%%%%%%%%%%%%%%%%%%
%%%%%%%%%%%%%%%%%%%%%%%%%%%%%%%%%%%%%%%%%%%%%%%%%%%%%%
\section{}

\begin{frame}[plain,allowframebreaks,noframenumbering]{Bibliografia}

\begin{thebibliography}{Pressman, 2001}

\bibitem[Pressman, 2001]{Pressman-2001}

Pressman, R.

\newblock{{\em ``Software Engineering: A Practioner's Approach"}. 4th edition. McGraw-Hill, 2001.}

\bibitem[PMI, 2008]{PMI-2008}

Project Management Institute, Inc.

\newblock{{\em ``A Guide To The Project Management Body Of Knowledge"}. 2008.}

\end{thebibliography}

\end{frame}

\begin{frame}[plain,noframenumbering]

\begin{center}
\includegraphics[keepaspectratio, width=.8\textwidth]{template/happycat-end}
\end{center}
\end{frame}

\end{document}